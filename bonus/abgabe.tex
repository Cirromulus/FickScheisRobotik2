\documentclass{../Vorlage/mat}
\lstset{
	basicstyle=\small
}

\begin{document}
\maketitle{Sebastian Bliefert}{}{Nils Drebing}{}{Pascal Pieper}{}{15.12.2016}{Bonus} \\

\section*{Aufgabe A}


\section*{Aufgabe B}


\section*{Aufgabe C}


\section*{Aufgabe D}
\subsection*{1}
\subsubsection*{(a)}
\begin{equation}
	R^{(1)} = \begin{pmatrix}
	0.87461971 & -0.4848092 & 0 \\
	0.4848092 & 0.87461971 & 0\\
	0&0&1
	\end{pmatrix}
\end{equation}
$q_0 = 1 + 0.87461971 + 0.87461971 + 1$\\
$q_1 = 0 - 0.4848092$\\
$q_2 = 0.4848092 - 0$\\
$q_3 = -0,4848092 - 0.4848092$\\
Also ist $\hat{q} = (3.74923942, -0.4848092, 0.4848092, -0.9696184)$
\subsubsection*{(b)}
$q_0 = \cos{\frac{\phi}{2}}$
\end{document}
