\documentclass{../Vorlage/mat}
\lstset{
	basicstyle=\small
}

\begin{document}
\maketitle{Sebastian Bliefert}{}{Nils Drebing}{}{Pascal Pieper}{}{8.12.2016}{2} \\

\tableofcontents
\clearpage

\section{abc}



\section*{Aufgabe 1}
%Erlautern Sie interne und externe Sensoren und geben Sie jeweils ein Beispiel. Erkl ¨ aren Sie wie ¨
%jedes Beispiel, das Sie gegeben haben, funktioniert. [20 Punkte]
Interne Sensoren messen den Zustand der Roboters (z.B. Wärmesensoren in Gelenken), externe Sensoren dagegen den Zustand der Welt um den Roboter (z.B. Laserscanner). Dabei kann dann noch unterschieden werden zwischen aktiven und passiven Sensoren. Passive messen nur die Welt um sich herum (oder auch den Zustand des Roboters falls es ein interner Sensor ist) (z.B. eine Kamera). Aktive Sensoren manipulieren die Welt (oder was auch immer sie messen wollen) zusätlich noch wie zum Beispiel Laserscanner, die einen Laserstrahl aussenden um Entfernungen zu messen.\\
\\
\textbf{Wärmesensoren} funktionieren mittels eines Leiters (Heiß- oder Kaltleiter) der bestromt wird. Dadurch, dass sich der Widerstand des Leiters mit der Temperatur des Leiters ändert kann an der resultierenden Stromstärke die Temperatur des Leiters ermittelt werden. Dies geschieht dann häufig softwareseitig, wo die Stromstärke in die Temperatur umgerechnet wird.\\
\\
\textbf{Laserscanner} senden einen Laserstrahl in eine Richtung aus und messen, wie lange es dauert bis dieser zurück kommt. Anhand dieser Messung kann mit Hilfe der Kenntnis über die Lichtgeschwindigkeit die Entfernung zu dem Punkt, von dem er reflektiert wurde ermittelt werden. Dies Passiert nun in hoher geschwindigkeit hintereinander und in verschiedene Richtungen, wodurch eine Punktwolke generiert werden kann, die die Welt um den Roboter abbildet.

\section*{Aufgabe 2}
%Diese Ubung soll mit Mars gemacht werden. Die Installation von Mars sollten Sie schon in der ¨
%Vorherige Ubung gemacht haben. ¨
%Unser reaktiver Robot hat zwei Kameras auf seinem Chassis montiert. Die Umgebung in der der
%Roboter ist, ist sehr simpel. Der Boden ist flach und die Beleuchtung ist auch fur die Aufgabe ¨
%geeignet.
%Irgendwo in der Nahe von dem Roboter liegen einige Kugeln. Die erste Aufgabe unseres Robot- ¨
%ers ist neben der rotten Kugel anzukommen. Um das zu schaffen werden wir einen Verhalten
%basierten Ansatz benutzen. Erstmal muss der Robot die Kugel sehen. Dann, muss er sich zu der
%Kugel annahren. Eine Point-Turn Drehung sollte den Roboter erlauben die Kugel zu finden. ¨
%Implementieren Sie ein Point-Turn Drehung Verhalten. [10 Punkte]
%Implementieren Sie ein Algorithmus das die Information aus dem Kamera Bild benutzt um die
%Kugel zu finden und zu ihr anzukommen. [20 Punkte]
%Wenn der Abstand zur Kugel ungefahr 1.0 Meter wird, soll der Roboter rum die Kugel fahren ¨
%ohne sie auf einem von dem Kamera Bilder zu verlieren. Wie schnell kann dein Roboter um die
%Kugel fahren? [30 Punkte]
%Die letzte Aufgabe des Roboters ist am schwierigsten. Der Roboter soll die verschiedenen
%Kugeln bis zur Grauen Wand tragen. Dafur m ¨ ussen andere Verhaltenweisen implementiert ¨
%werden. [10 Extra Punkte]

\subsection*{Point Turn}
Mit dieser Methode kann der Roboter sich auf der Stelle drehen. Die Geschwindigkeit wird aus einer globalen Variable \textit{speed} gezogen.
\lstinputlisting[linerange={26-30}]{python/behavior.py}

\subsection*{Kamerabildalgorithmus}
Zur Erkennung des roten Balls wird nach roten Pixeln mit entsprechenden RGB-Werten gesucht. Dabei wurden die Parameter durch \textit{trial and error} ermittelt. Der Algorithmus endet, sobald ein einziges rotes Pixel im Bild gefunden wurde. Als Rückgabe wird ein Tupel zurückgegeben, das zum einen angibt, in welchem Kamerabild das Pixel gefunden wurde. Zum anderen wird die vertikale Position des Pixels im Bild zurückgegeben, wobei die Nullkoordinate in der Mitte des Bildes liegt.
\lstinputlisting[linerange={57-83}]{python/behavior.py}

Je nach dem, wo sich der Ball im Bild befindet (tendenziell links, tendenziell rechts), wird der Motorwert  gewichtet angepasst. Auf diese Weise wird der Ball in der Mitte der rechten Kamera gehalten.
\subsection*{Ballfindung}
\lstinputlisting[linerange={33-41}]{python/behavior.py}

\subsection*{Kugel umrunden}
Hier soll die Kugel im Zentrum des Bildes der linken Kamera behalten. Die Motorwerte werden je nach Position des Balls im Bild angepasst, sodass eine kreisförmige Trajektorie abgefahren wird.
\lstinputlisting[linerange={86-110}]{python/behavior.py}

\subsection*{Kugeln an die Wand}
Nicht implementiert.

\subsection*{Statemachine}
\lstinputlisting[linerange={113-144}]{python/behavior.py}

\section*{Aufgabe 3}
%Es gibt dann 3 verschiedene reaktive Verhalten hier erlautert: ¨
%• Kugel Finden.
%• In der Nahe von der Kugel anzukommen. ¨
%• Rum Die Kugel zu drehen.
%1
%Ihre Implementation soll die Verschiedene Verhalten in einer organisierten Architektur zusammenfassen. Um zu analisieren wie geeignet eine reaktive Architektur ist, muss man sich untern anderen folgende Fragen stellen: wie einfach ist eine neus Verhalten zu integrieren (z.B.
%Hindernis vermeiden)? Sind die verschiedene Verhalten unabhangig von einander? Wenn ein ¨
%Verhalten sich geandert w ¨ urde, w ¨ urden die anderen auch ¨ Anderungen brauchen? Wie schnell ¨
%kann der Roboter erkennen, dass das Verhalten geandert werden muss? ¨
%Bitte erklaren und verteidigen Sie ihre implementierte Architektur. [10 Punkte]
Unsere Architektur benutzt eine Statemachine für die Verwaltung der Zustände.
Die Kugel wird gefunden, indem der Roboter sich dreht, bis er eine rote Kugel (bzw. Rote Pixel) sieht. Dann dreht er sich (schon leicht fahrend) zu der Kugel, bis sie sich in der Mitte seiner Frontkamera befindet. Dann fährt er geradeaus. Beim Fahren korrigiert er seine Richtung leicht in Abhängigkeit der Verschiebung der Kugel. Wenn die Kugel groß genug ist, wechselt er in den Umrundungsmodus. Dort dreht er sich, bis die Kugel in seiner Seitenkamera auftaucht. Dann fährt er geradeaus und dreht sich gleichzeitig in Abhängigkeit der Position der Kugel. Also fährt der Roboter geradeaus, wenn die Kugel in der Mitte seiner Seitenkamera ist; dreht anteilig links, wenn die Kugel links verschoben ist, und anders herum. Wenn die Kugel aus dem Sichtfeld verschwindet, geht er wieder in den Findungszustand.\\
Diese Zustände sind unabhängig voneinander. Alle Verhalten sind in sich geschlossen. States können durch eine \textit{if}-Klausel in der \textit{execute()}-Funktion hinzugefügt werden. \\
Die Änderung der Zustände können in jedem Durchlauf einer Bildverarbeitung (\texttt{execute()}) gemacht werden. Je mehr Kamerabilder pro Sekunde berechnet werden, desto schneller kann der Roboter um die Kugel kreisen.
blkabvalbalbalba

blkabvalbalbalba

blkabvalbalbalba

blkabvalbalbalba

blkabvalbalbalba

blkabvalbalbalba

blkabvalbalbalba

blkabvalbalbalba

blkabvalbalbalba

blkabvalbalbalba

blkabvalbalbalba

blkabvalbalbalba

blkabvalbalbalba

blkabvalbalbalba

blkabvalbalbalba

blkabvalbalbalba

blkabvalbalbalba

blkabvalbalbalba

blkabvalbalbalba

blkabvalbalbalba

blkabvalbalbalba

blkabvalbalbalba

blkabvalbalbalba

blkabvalbalbalba

blkabvalbalbalba

blkabvalbalbalba

blkabvalbalbalba

blkabvalbalbalba

blkabvalbalbalba

blkabval
\section{abcd}
\end{document}
