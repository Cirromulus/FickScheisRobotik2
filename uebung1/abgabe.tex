\documentclass{./Vorlage/mat}

\begin{document}
\maketitle{Sebastian Bliefert}{}{Nils Drebing}{}{Pascal Pieper}{}{09.11.2016}{1} \\

\section*{Aufgabe 1}

\section*{Aufgabe 2}
Die Fusion der Sensorwerte wird mittels einer Maximumfunktion errechnet. Alternativ ist auch der Durchschnitt möglich gewesen, allerdings hätte auf diese Art die Ausgabe eine andere Range gehabt.
\subsection*{a) - Liebe}
\begin{lstlisting}
def love(light):
    global right_actuator
    global left_actuator
    sensors = sensorFeedback(light)
    logMessage(str(sensors))
    left_actuator  = (.9 - sensors[0]) * (.9 - sensors[0]) * 10
    right_actuator = (.9 - sensors[1]) * (.9     - sensors[1]) * 10
    logMessage(str(left_actuator) + ":" + str(right_actuator))
\end{lstlisting}

\subsection*{b) - Aggression}
\begin{lstlisting}
def hate(light):
   global right_actuator
   global left_actuator
   sensors = sensorFeedback(light)
   logMessage(str(sensors))
   left_actuator  = 10 + ((sensors[1]) * 25) * ((sensors[1]) * 25)
   right_actuator = 10 + ((sensors[0]) * 25) * ((sensors[0]) * 25)
   logMessage(str(left_actuator) + ":" + str(right_actuator))
\end{lstlisting}

\subsection*{c) - Angst}
\begin{lstlisting}
def feeeeear(light):
    global right_actuator
    global left_actuator
    sensors = sensorFeedback(light)
    logMessage(str(sensors))
    left_actuator  = 1 - 10 * sensors[1]
    right_actuator = 1 - 10 * sensors[0]
    logMessage(str(left_actuator) + ":" + str(right_actuator))  
\end{lstlisting}


\subsection*{d) - Neugier}
\begin{lstlisting}
def neugear(light):
    global right_actuator
    global left_actuator
    sensors = sensorFeedback(light)
    logMessage(str(sensors))
    left_actuator  = 5 + 10 * (1 - sensors[1])
    right_actuator = 5 + 10 * (1 - sensors[0])
    logMessage(str(left_actuator) + ":" + str(right_actuator))
\end{lstlisting}


\section*{Aufgabe arsch}


\end{document}
\grid
