\documentclass{./Vorlage/mat}

\begin{document}
\maketitle{Sebastian Bliefert}{}{Nils Drebing}{}{Pascal Pieper}{}{09.11.2016}{1} \\

\section*{Aufgabe 1}
\subsection*{a}
Grey Walters Schildkröte verwendet zwei Sensoren, zwei Aktuatoren und zwei Vakuumröhren als "`Nervenzellen'". Bei der Sensorik handelt es sich um einen Kontaktsensor und einen Lichtsensor, die Informationen über die Umgebung zur Verfügung stellen. Zum Antrieb wird ein einziges Vorderrad verwendet, dass durch jeweils einen Motor gedreht sowie angetrieben wird.\\
Der Lichtsensor ist mit der Rotationsachse des Antriebsrads verbunden, sodass immer das Licht in Fahrtrichtung registriert wird.
\newpage

\subsection*{b}
\underline{Eingabeparameter:}
\begin{itemize}
\item Licht \textbf{L}: Wert, der vom Lichtsensor eingefangen wird.
\item Licht-Threshold \textbf{t}: Lichtwert, der bestimmt, ab wann eine Lichtquelle als \textit{stark} bezeichnet werden kann.
\item Batteriezustand \textbf{b}: Gibt an, ob Batterie geladen (Wert = 1) oder beinahe leer (Wert = 0) ist.
\item Bumpsensor \textbf{obs}: 1, falls der Bumpsensor aktiviert wurde, sonst 0.
\end{itemize}
\begin{tikzpicture}[shorten >=1pt,node distance=5cm,on grid,auto] 
   \node[state,initial] (q_0)   {seek light}; 
   \node[state] (q_1) [below=of q_0]  {turn/push}; 
   \node[state] (q_2) [left=of q_0]  {head light};
   \node[state] (q_3) [right=of q_0]  {back away};
   \node[state] (q_4) [above=of q_0]  {recharge}; 
   
   \path[->]
   (q_0) edge [bend left] node {obs = 1} (q_1)
   (q_0) edge [bend left]node [above]{0 < L < t} (q_2)
   (q_0) edge [bend left]  node [below]{t < L} (q_3)
   (q_0) edge [bend left = 45] node {b = 0} (q_4)
   
   (q_1) edge [bend left] node {obs = 0} (q_0)
   (q_1) edge [bend left, out=90,in=90,bend angle=180,looseness=2] node {b = 0} (q_4)
   
   (q_2) edge [bend left] node [below]{L = 0} (q_0)
   (q_2) edge [bend right]node {obs = 1} (q_1)
   (q_2) edge [bend right=40] node [above]{t < L} (q_3)
   (q_2) edge [bend left]  node {b = 0} (q_4)
   
   (q_3) edge [bend left] node [above]{L = 0} (q_0)
   (q_3) edge [bend left] node {obs = 1} (q_1)
   (q_3) edge [bend right=40] node [below]{0 < L < t} (q_2)   
   (q_3) edge [bend right] node [right]{b = 0} (q_4)
   
   (q_4) edge [bend left = 45] node {b = 1} (q_0)
   ;
\end{tikzpicture}

\section*{Aufgabe 2}
Die Fusion der Sensorwerte wird mittels einer Maximumfunktion errechnet. Alternativ ist auch der Durchschnitt möglich gewesen, allerdings hätte auf diese Art die Ausgabe eine andere Range gehabt.
\subsection*{a) - Liebe}
\begin{lstlisting}
def love(light):
    global right_actuator
    global left_actuator
    sensors = sensorFeedback(light)
    logMessage(str(sensors))
    left_actuator  = (0.1/(sensors[0]+0.09))-0.112
    right_actuator = (0.1/(sensors[1]+0.09))-0.112
    logMessage(str(left_actuator) + ":" + str(right_actuator))
\end{lstlisting}

Die Skalierungswerte wurden so berechnet, dass ohne Licht etwa 1 und ab einem Sensor-Wert von etwa 0.8 0 heraus kommt.

\subsection*{b) - Aggression}
\begin{lstlisting}
def hate(light):
   global right_actuator
   global left_actuator
   sensors = sensorFeedback(light)
   logMessage(str(sensors))
   left_actuator  = 1 + (sensors[0] * sensors[0])
   right_actuator = 1 + (sensors[1] * sensors[1])
   logMessage(str(left_actuator) + ":" + str(right_actuator))
\end{lstlisting}

\subsection*{c) - Angst}
\begin{lstlisting}
def fear(light):
    global right_actuator
    global left_actuator
    sensors = sensorFeedback(light)
    logMessage(str(sensors))
    left_actuator  = 0.5 * (sensors[0]-sensors[1]+1.2) * (sensors[0]-sensors[1]+1.2) * (sensors[0]-sensors[1]+1.2) * (sensors[0]-sensors[1]+1.2)
    right_actuator = 0.5 * (sensors[1]-sensors[0]+1.2) * (sensors[1]-sensors[0]+1.2) * (sensors[1]-sensors[0]+1.2) * (sensors[1]-sensors[0]+1.2)
    logMessage(str(left_actuator) + ":" + str(right_actuator))  
\end{lstlisting}

Die Funktion für die Aktuatoren entspricht $ 0.5 (diff+1.2)^4$ (wobei $diff$ der Differenz zwischen den beiden Sensoren entspricht) was dazu führt, dass niemals einer der Aktuatoren einen negativen Wert übergeben bekommt, bei einer differenz von 0 die Aktuatoren den Wert $1$ bekommen und dennoch sehr schnell auf Unterschiede zwischen den Sensoren reagiert werden kann.

\subsection*{d) - Neugier}
\begin{lstlisting}
def curiosity(light):
    global right_actuator
    global left_actuator
    sensors = sensorFeedback(light)
    logMessage(str(sensors))
    left_actuator  = 5 + 10 * (1 - sensors[1])
    right_actuator = 5 + 10 * (1 - sensors[0])
    logMessage(str(left_actuator) + ":" + str(right_actuator))
\end{lstlisting}

\section*{Aufgabe 3}

Grey Walters Schidkröte wurde von uns folgendermaßen implementiert:

\begin{lstlisting}
def tortoise(light, distance):
	sensors = sensorFeedback(light)
	schwellwert_love = 0.4
	logMessage("tortoise: " + str(sensors[0]) + ":" + str(sensors[1]))
	if sensors[0] == 0 and sensors[1] == 0:
		randomWalk(distance)
	else:
		if sensors[0] < schwellwert_love and sensors[1] < schwellwert_love:
			love(light)
		else:
			fear(light)
\end{lstlisting}

Das Verhalten entspricht leider nicht ganz dem der eigentlichen Schildkröte. Dies ist allerdings unter den gegebenen Umständen nicht besser lösbar, da keine Variablen über zwei Durchläufe mitgegeben werden können.
Wäre dies möglich könnte mit helfer von \texttt{booleans} nach dem erreichen des Schwellwertes \texttt{fear} so lange ausgeführt werden, bis kein Sensor mehr Licht empfängt und dann wieder von vorne begonnen werden.

\end{document}
\grid
